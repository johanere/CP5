%%
%% Automatically generated file from DocOnce source
%% (https://github.com/hplgit/doconce/)
%%
%%


%-------------------- begin preamble ----------------------

\documentclass[%
oneside,                 % oneside: electronic viewing, twoside: printing
final,                   % draft: marks overfull hboxes, figures with paths
10pt]{article}

\listfiles               %  print all files needed to compile this document


\usepackage[totoc]{idxlayout}   % for index in the toc
\usepackage[nottoc]{tocbibind}  % for references/bibliography in the toc

\usepackage{relsize,makeidx,color,setspace,amsmath,amsfonts,amssymb}
\usepackage[table]{xcolor}
\usepackage{bm,ltablex,microtype}
\usepackage{comment} 
\usepackage[pdftex]{graphicx}

\usepackage{fancyvrb} % packages needed for verbatim environments

\usepackage[T1]{fontenc}
%\usepackage[latin1]{inputenc}
\usepackage{ucs}
\usepackage[utf8x]{inputenc}

\usepackage{lmodern}         % Latin Modern fonts derived from Computer Modern


\usepackage{pgfplotstable, booktabs}

\pgfplotstableset{
    every head row/.style={before row=\toprule,after row=\midrule},
    every last row/.style={after row=\bottomrule}
}





% Hyperlinks in PDF:
\definecolor{linkcolor}{rgb}{0,0,0.4}
\usepackage{hyperref}
\hypersetup{
    breaklinks=true,
    colorlinks=true,
    linkcolor=linkcolor,
    urlcolor=linkcolor,
    citecolor=black,
    filecolor=black,
    %filecolor=blue,
    pdfmenubar=true,
    pdftoolbar=true,
    bookmarksdepth=3   % Uncomment (and tweak) for PDF bookmarks with more levels than the TOC
    }
%\hyperbaseurl{}   % hyperlinks are relative to this root

\setcounter{tocdepth}{2}  % levels in table of contents

% --- fancyhdr package for fancy headers ---
\usepackage{fancyhdr}
\fancyhf{} % sets both header and footer to nothing
\renewcommand{\headrulewidth}{0pt}
\fancyfoot[LE,RO]{\thepage}
% Ensure copyright on titlepage (article style) and chapter pages (book style)
\fancypagestyle{plain}{
  \fancyhf{}
  \fancyfoot[C]{{\footnotesize \copyright\ 1999-2018, "Computational Physics I FYS3150/FYS4150":"http://www.uio.no/studier/emner/matnat/fys/FYS3150/index-eng.html". Released under CC Attribution-NonCommercial 4.0 license}}
%  \renewcommand{\footrulewidth}{0mm}
  \renewcommand{\headrulewidth}{0mm}
}
% Ensure copyright on titlepages with \thispagestyle{empty}
\fancypagestyle{empty}{
  \fancyhf{}
  \fancyfoot[C]{{ }}
  \renewcommand{\footrulewidth}{0mm}
  \renewcommand{\headrulewidth}{0mm}
}

\pagestyle{fancy}


% prevent orhpans and widows
\clubpenalty = 10000
\widowpenalty = 10000

% --- end of standard preamble for documents ---


% insert custom LaTeX commands...

\raggedbottom
\makeindex
\usepackage[totoc]{idxlayout}   % for index in the toc
\usepackage[nottoc]{tocbibind}  % for references/bibliography in the toc
\usepackage{listings}
\usepackage[normalem]{ulem} 	%for tables
\useunder{\uline}{\ul}{}
\usepackage{hyperref}
\usepackage[section]{placeins} %force figs in section

%-------------------- end preamble ----------------------

\begin{document}

% matching end for #ifdef PREAMBLE

\newcommand{\exercisesection}[1]{\subsection*{#1}}


% ------------------- main content ----------------------



% ----------------- title -------------------------

\thispagestyle{empty}

\begin{center}
{\LARGE\bf
\begin{spacing}{1.25}
Simulating the Ising model using the Metropolis algorithm
\end{spacing}
}
\end{center}

% ----------------- author(s) -------------------------

\begin{center}
{\bf Johan Nereng}
\end{center}

    \begin{center}
% List of all institutions:
\centerline{{\small Department of Physics, University of Oslo, Norway}}
\end{center}
    
% ----------------- end author(s) -------------------------

% --- begin date ---
\begin{center}
Nov 21, 2018
\end{center}
% --- end date ---

\vspace{1cm}

\section{Abstract}
\section{Introduction}

For the purpose of approximating the derivatives of $u$, three methods are applied: explicit forward Euler, implicit Backward Euler, and the implicit Crank-Nicolson with a time-centered scheme.
\section{Methods}

\begin{align}
\nabla^2u(x,t)=\frac{\partial u(x,t)}{\partial t} \\
u_xx=u_t
\end{align}
Initial conditions, boundry conditions \newline

Before approximating the derivatives, the problem discretized. The spacial domain , $x \in [0,L]$, $L=1$, is discretized over $n+1$ grid points, such that $x_i=i \Delta x$, $i=0,1,...,n,n+1$, where $\Delta x= \frac{1}{n+1}$. Time, $t=[0,T]$, is is expressed as $t_j=j \Delta t$, $j\geq 0$, where $\Delta t=\frac{1}{T}$. The dizcrete approximation of the function $u$ is then defined as $u(x_i,t_j)=u_{i,j}$, with boundary conditions (B.C) $u_{0,j}=0$ and $u_{n+1,j}=1$. \newline

The solution to the problem is assumed to be $u$, and in order to dimdum, a change of variables from $u$ to $v$ is applied, where $v$ is discretized similarly to $x$: $v(x,t)=v_{i,j}$.  $u_{i,j}=v_{i,j}+x \implies v_{i,j}=u_{i,j}-x_i$. This changes the B.C to \begin{align*}
v_{0,j}=v_{1,j}=0
\end{align*}
Recalling the initial conditions, $u_{i,0}=0$, for $i<L$, $ \implies v_{i,0}=u_{i,0}-x_i$

\textbf{bytt med v}
Approximating derivatives:
Explicit forward Euler (assignment paper/lecture notes):
\begin{align}
u_t \approx \frac{u(x_i,t_j+\Delta t) -u(x_i,t_j)}{\Delta t} \\
u_{xx} \approx \frac{u(x_i+\Delta x,t_j) -2u(x_i,t_j)+u(x_i-\Delta x,t_j)}{\Delta x^2}\\
\end{align}
Applying compact notation introduced in the discretization:
\begin{align}
u_t \approx \frac{u_{i,j+1} -u_{i,j}}{\Delta t} \\
u_{xx} \approx \frac{u_{i+1,j} -2u_{i,j}+u_{i-1,j}}{\Delta x^2}\\
\end{align}
Where $u_t$ has local truncation error (LTE) $O(\Delta t)$ and  $u_xx$ has $O(\Delta x^2)$. \newline

Setting $u_t=u_{xx}$, defining $\alpha = \Delta t/\Delta x^2$, and solving for $v_{i,j+1}$ yields what is known as the explicit scheme (ref lec notes): 
\begin{equation}
v_{i,j+1}=\alpha v_{i-1,j}+(1-2 \alpha)v_{i,j}+\alpha v_{i+1,j}
\end{equation}
As $v(0,j)=v(n+1,j)=0$ for all $j$, the time dependent part of the solution is now expressed as:
\begin{equation} V_j=
\begin{bmatrix}
v_{1,i} \\ v_{2,i} \\ ...\\ v_{n,i}
\end{bmatrix}
\end{equation}


Similarly, implicit backward euler:
\begin{align}
u_t \approx \frac{u_{i,j} -u_{i,j-1}}{\Delta t} \\
u_{xx} \approx \frac{u_{i+1,j} -2u_{i,j}+u_{i-1,j}}{\Delta x^2}\\
\end{align}


\section{Algorithms}
\label{sec:M.Algo}


\end{document}





% ------------------- end of main content ---------------



